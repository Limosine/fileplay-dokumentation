\documentclass[a4paper]{article}
\usepackage{graphicx}
\usepackage{pdfpages}
\usepackage{hyperref}
\usepackage{etoc}
\usepackage[ngerman]{babel}
\usepackage[T1]{fontenc}

% cSpell:words Filesharing techstack Dezhong Zhuang

\begin{document}


\begin{titlepage}
  \begin{center}
    \textbf{Ernst-Mach-Gymnasium Haar}

    \vspace{0.5cm}

    Ausbildungsabschnitt 11/2

    \vspace{1cm}

    \includegraphics[width=0.2\textwidth]{fileplay.png}

    \vspace*{1cm}

    \textbf{Fileplay}

    \vspace{0.5cm}

    Dateiübertragungswebseite mit Kontaktfunktion

    \vfill
           
    \textbf{
      \textit{Quentin Frey, Leonhard Masche, Dezhong Zhuang}
    }

  \end{center}
\end{titlepage}

\setcounter{tocdepth}{2}
\tableofcontents
\newpage

\section{Einleitung}
Unser Projekt ist eine Filesharing App (Webseite), mit der Dateien Peer-to-Peer,
also von Nutzer zu Nutzer, ohne zwischengeschalteten Server*, übertragen werden
können. Die App basiert auf modernen Web-Technologien, wie z.B.
\href{https://web.dev/push-notifications-web-push-protocol/}{Web Push},
\href{https://web.dev/learn/pwa/service-workers/}{Service Workers} und
\href{https://web.dev/progressive-web-apps/}{Progressive Web Apps}. Durch die
Design-Philosophie des
\href{https://developer.mozilla.org/en-US/docs/Glossary/Progressive_Enhancement}{Progressive
  Enhancement} werden aber auch ältere Browser unterstützt. Eine Besonderheit der
App ist es, dass sie dauerhafte Verbindungen durch Kontakte möglich macht.
Kontakte können zwischen mehreren Nutzern geknüpft werden, wobei jeder Nutzer
mehrere Geräte haben kann. Zum Schutz der Privatsphäre und Benutzersicherheit
werden keine identifizierbaren Informationen auf dem Server (in der Datenbank)
gespeichert und die Dateien im Transit Ende-zu-Ende verschlüsselt. Die
Motivation für dieses Projekt war das Problem, dass es keine einfache
Möglichkeit gibt, zwischen iOS / Android / Windows / Linux Geräten Dateien zu
teilen. Bestehende Webseiten nutzen hierfür einen Link, den der Empfänger öffnen
muss, um die Datei zu empfangen. Hierbei stellt sich dann wieder die Frage, wie
man diesen Link teilt. Um dieses Problem zu behandeln, implementiert die App ein
Kontakt-System, in dem sich Nutzer einmalig über einen 6-stelligen Code
verbinden und von nun an einfach miteinander Dateien teilen können. Dieses
System ist vor allem in einer Umgebung nützlich, in der man oft Dokumente mit
denselben Personen teilt, wie es z.B. in der Schule der Fall ist.

\section{Dokumentation}

\subsection{Implementierung}
Die App ist in Form einer Webseite realisiert, und ist somit auf allen Geräten
zugänglich, die einen Browser haben. Für die Entwicklung haben wir uns für das
Framework \href{https://kit.svelte.dev/}{SvelteKit} im NodeJS Ökosystem
entschieden. Dadurch muss die App nicht manuell in den Browser-kompatiblen
Sprachen geschrieben werden, sondern kann in einer modernen Sprache (Svelte in
Verbindung mit Typescript) geschrieben werden, wodurch die Entwicklung durch
vorgefertigte Implementierungsmuster und Tools stark vereinfacht wird. Die App
ist nach dem Muster einer Single-Page-Application aufgebaut, d.h. die Seite wird
nur einmal geladen, und alle weiteren Interaktionen finden über JavaScript
statt. Dadurch wird die App sehr schnell, da nur die Daten, die benötigt werden,
vom Server geladen werden müssen. Zusätzlich wird die App dadurch sehr
benutzerfreundlich, da der Nutzer nicht durch das Laden von verschiedenen Seiten
unterbrochen wird. Durch die optionale Installation als Progressive Web App
(PWA) kann die App auch auf dem Homescreen installiert werden, und verhält sich
dann wie eine native App.\\
Die Entwicklung der App findet in einem öffentlichen GitHub Repository statt.
Dieses Repository ist unter \url{https://github.com/leonhma/fileplay} zu finden,
und wird automatisch durch Integration mit Cloudflare auf den Adressen
\url{https://app.fileplay.me} (für den {main}-branch) und
\url{https://dev.fileplay.pages.dev} (für den {dev}-branch) verfügbar gemacht.\\
Der Programmcode für einen Microservice der die Kommunikation von Server zu
Client verwaltet ist unter \\\url{https://github.com/leonhma/fileplay-worker} zu
finden. Dieser Mikroservice nutzt APIs die nicht in SvelteKit verfügbar sind,
und wird deshalb separat gehosted.

\subsection{Architektur}
Die App ist nach dem MVC-Modell in drei Teile aufgeteilt: Frontend, Backend und
Datenbank. Das Frontend (View) ist die Webseite, die der Nutzer sieht und mit
der er interagiert. Das Backend (Controller) ist der Server, der die Webseite
ausliefert und die Anwendungslogik verwaltet, und die Datenbank ist das Modell,
in dem die Daten gespeichert werden. Sowohl das Frontend, als auch große Teile
des Backends werden vom Framework SvelteKit verwaltet. Dieses Framework erlaubt
es, die Webseite mit einem angepassten Syntax zu schreiben und
Entwickler-freundlich in Komponenten zu zerteilen. Zusätzlich bietet es eine
einfache Möglichkeit, Backend und Frontend Seite an Seite zu entwickeln.
Zusätzlich zu dem Backend-Code der in SvelteKit geschrieben ist, gibt es noch
einen Microservice, der als \href{https://workers.cloudflare.com/}{Cloudflare
  Worker} in Typescript geschrieben ist. Dieser Verwaltet die WebSockets, die für
die Kommunikation von Server zu Client verwendet werden. Die Datenbank ist eine
SQLite Datenbank, die mit dem Backend verbunden ist und Daten wie Geräte, Nutzer
und Kontakte speichert.\\
Die einzelnen Teile der App (Backend, Datenbank und Microservice) werden alle
auf Cloudflare gehosted. Cloudflare bietet verschieden Web-Hosting services an,
darunter auch Pages, welches über eine GitHub Integration einfach mit dem
Build-output von SvelteKit verbunden werden kann. Zusätzlich wird eine SQLite
Datenbank namens D1 benutzt, welche im Backend einfach durch Umgebungsvariablen
eingebunden werden kann. Der Microservice ist ein Cloudflare Worker, welcher
mithilfe der neuen WebSocket Hibernation API ressourcenschonend die Verbindungen
von Server zu Client verwaltet.\\
Die nutzerseitige Logik der App findet vor allem in form von JavaScript
clientseitig statt. Wenn Daten vom Server abgefragt werden müssen (so zum
beispiel die liste der Kontakte), wird dies über eine REST API gemacht. Die
Kommunikation von Server zu Client findet über WebSockets statt. Alternativ wird
hier auch eine Implementation über Web Push verwendet, jedoch wurde diese
aufgrund von Kompatibilitätsproblemen mit Edge und Firefox kurzfristig
deaktiviert.

\subsection{Geräte, Nutzer, Kontakte}

Um eine einfache Organisation und Nutzung der App zu ermöglichen, werden
Geräte, Nutzer und Kontakte unterschieden. Ein Gerät ist ein Gerät, auf dem die
App installiert ist. Ein Nutzer ist eine Person, die die App nutzt. Ein Nutzer
kann mehrere Geräte haben, und ein Gerät kann nur einen. Ein Kontakt
repräsentiert eine Verbindung zwischen zwei Nutzern. Ein Nutzer kann mehrere
Kontakte haben.


\subsection{Filesharing}
Wenn ein Nutzer eine Datei teilen möchte, wählt er einen verfügbaren Kontakt
aus, dessen ID dann in einer Anfrage an den Server gesendet wird. Dieser schickt dann die Anfrage
per WebSocket (oder Web Push) an die einzelnen Geräte des Kontakts. Wenn ein Gerät die Anfrage
akzeptiert, schickt dieses Gerät eine Antwort an den Server, mit der PeerJs Adresse, an die die Datei übertragen
werden soll. Diese Antwort enthält auch den Public Key des Geräts, der für die
Verschlüsselung der Datei verwendet wird. Der Server leitet diese Antwort dann
an den ursprünglichen Nutzer weiter, der dann die Datei an die PeerJs Adresse
senden kann.



\subsection{User-Story}

...




\section{Aufteilung der Aufgaben}

\subsection*{Mitglieder}
Dezhong Zhuang, Quentin Frey, Leonhard Masche

\section{To-Do}
mirror to readme\\
Anleitung\\
techstack\\
Webtechnologien und Fallbacks (version graphics)\\
mvc muster\\
anschaulich! (Grafiken, Diagramme, ...)


\end{document}