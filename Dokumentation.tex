\documentclass[a4paper]{article}
\usepackage{graphicx}
\usepackage{pdfpages}
\usepackage{hyperref}
\usepackage{etoc}
\usepackage[ngerman]{babel}
\usepackage[T1]{fontenc}

% cSpell:words Filesharing techstack Dezhong Zhuang

\begin{document}


\begin{titlepage}
  \begin{center}
    \textbf{Ernst-Mach-Gymnasium Haar}

    \vspace{0.5cm}

    Ausbildungsabschnitt 11/2

    \vspace{1cm}

    \includegraphics[width=0.2\textwidth]{fileplay.png}

    \vspace*{1cm}

    \textbf{Fileplay}

    \vspace{0.5cm}

    Dateiübertragungswebseite mit Kontaktfunktion

    \vfill
           
    \textbf{
      \textit{Quentin Frey, Leonhard Masche, Dezhong Zhuang}
    }

  \end{center}
\end{titlepage}

\setcounter{tocdepth}{2}
\tableofcontents
\newpage

\section{Einleitung}
Unser Projekt ist eine Filesharing App (Webseite), mit der Dateien Peer-to-Peer,
also von Nutzer zu Nutzer, ohne zwischengeschalteten Server*, übertragen werden
können. Die App basiert auf modernen Web-Technologien, wie z.B.
\href{https://web.dev/push-notifications-web-push-protocol/}{Web Push},
\href{https://web.dev/learn/pwa/service-workers/}{Service Workers} und
\href{https://web.dev/progressive-web-apps/}{Progressive Web Apps}. Durch die
Design-Philosophie des
\href{https://developer.mozilla.org/en-US/docs/Glossary/Progressive_Enhancement}{Progressive
  Enhancement} werden aber auch ältere Browser unterstützt. Eine Besonderheit der
App ist es, dass sie dauerhafte Verbindungen durch Kontakte möglich macht.
Kontakte können zwischen mehreren Nutzern geknüpft werden, wobei jeder Nutzer
mehrere Geräte haben kann. Zum Schutz der Privatsphäre und Benutzersicherheit
werden keine identifizierbaren Informationen auf dem Server (in der Datenbank)
gespeichert und die Dateien im Transit Ende-zu-Ende verschlüsselt. Die
Motivation für dieses Projekt war das Problem, dass es keine einfache
Möglichkeit gibt, zwischen iOS / Android / Windows / Linux Geräten Dateien zu
teilen. Bestehende Webseiten nutzen hierfür einen Link, den der Empfänger öffnen
muss, um die Datei zu empfangen. Hierbei stellt sich dann wieder die Frage, wie
man diesen Link teilt. Um dieses Problem zu behandeln, implementiert die App ein
Kontakt-System, in dem sich Nutzer einmalig über einen 6-stelligen Code
verbinden und von nun an einfach miteinander Dateien teilen können. Dieses
System ist vor allem in einer Umgebung nützlich, in der man oft Dokumente mit
denselben Personen teilt, wie es z.B. in der Schule der Fall ist.

\section{Dokumentation}

\subsection{Implementierung}
Die App ist in Form einer Webseite realisiert, und ist somit auf allen Geräten
zugänglich, die einen Browser haben. Für die Entwicklung haben wir uns für das
Framework \href{https://kit.svelte.dev/}{SvelteKit} im NodeJS Ökosystem
entschieden. Dadurch muss die App nicht manuell in den Browser-kompatiblen
Sprachen geschrieben werden, sondern kann in einer modernen Sprache (Svelte in
Verbindung mit Typescript) geschrieben werden, wodurch die Entwicklung durch
vorgefertigte Implementierungsmuster und Tools stark vereinfacht wird. Die
Entwicklung der App findet in einem öffentlichen GitHub Repository statt. Dieses
Repository ist unter \url{https://github.com/leonhma/fileplay} zu finden, und
wird automatisch durch Integration mit Cloudflare auf den Adressen
\url{https://app.fileplay.me} (für den {main}-branch) und
\url{https://dev.fileplay.pages.dev} (für den {dev}-branch) verfügbar gemacht.\\
Der Programmcode für einen Microservice ist unter
\\\url{https://github.com/leonhma/fileplay-worker} zu finden.

\subsection{Infrastruktur}

Das Cloud-Hosting wird von Cloudflare betrieben. Cloudflare 

\subsection{Architektur}
Die App ist nach dem MVC-Modell in drei Teile aufgeteilt: Frontend, Backend und
Datenbank. Das Frontend (View) ist die Webseite, die der Nutzer sieht und mit
der er interagiert. Das Backend (Controller) ist der Server, der die Webseite
ausliefert und die Anwendungslogik verwaltet, und die Datenbank ist das Modell,
in dem die Daten gespeichert werden. Sowohl das Frontend, als auch große Teile
des Backends werden vom Framework SvelteKit verwaltet. Dieses Framework erlaubt
es, die Webseite mit einem angepassten Syntax zu schreiben und
Entwickler-freundlich in Komponenten zu zerteilen. Zusätzlich bietet es eine
einfache Möglichkeit, Backend und Frontend Seite an Seite zu entwickeln.
Zusätzlich zu dem Backend-Code der in SvelteKit geschrieben ist, gibt es noch
einen Microservice, der als \href{https://workers.cloudflare.com/}{Cloudflare
  Worker} in Typescript geschrieben ist. Dieser Verwaltet die WebSockets, die für die Kommunikation
von Server zu Client verwendet werden. Die Datenbank ist eine SQLite Datenbank,
die mit dem Backend verbunden ist und Daten wie Geräte, Nutzer und Kontakte speichert.\\
todo Kommunikation der teile der app


\subsection{User-Story}

...




\section{Aufteilung der Aufgaben}

\subsection*{Mitglieder}
Dezhong Zhuang, Quentin Frey, Leonhard Masche

\section{To-Do}
mirror to readme\\
Anleitung\\
techstack\\
Webtechnologien und Fallbacks (version graphics)\\
mvc muster\\
anschaulich! (Grafiken, Diagramme, ...)


\end{document}